\section{Editing programs}
Vim has various commands that aid in writing computer programs.
Compile a program and directly jump to reported errors.
Automatically set the indent for many languages and format comments.
\localtableofcontentswithrelativedepth{+1}
\subsection{Compiling}
Vim has a set of so called "\verb!quickfix!" commands.
They enable you to compile a program from within Vim and then go through the errors generated and fix them (hopefully).
You can then recompile and fix any new errors that are found until finally your program compiles without any error.

The following command runs the program "\verb!make!" (supplying it with any argument you give) and captures the results:

\begin{Verbatim}[samepage=true]
 :make {arguments}
\end{Verbatim}

If errors were generated, they are captured and the editor positions you where the first error occurred.

Take a look at an example "\verb!:make!" session.
(Typical :make sessions generate far more errors and fewer stupid ones.)  After typing "\verb!:make!" the screen looks like this:

\begin{Verbatim}[samepage=true]
    :!make | &tee /tmp/vim215953.err 
    gcc -g -Wall -o prog main.c sub.c 
    main.c: In function 'main': 
    main.c:6: too many arguments to function 'do_sub' 
    main.c: At top level: 
    main.c:10: parse error before '}' 
    make: *** [prog] Error 1 

    2 returned 
    "main.c" 11L, 111C 
    (3 of 6): too many arguments to function 'do_sub' 
    Press ENTER or type command to continue 
\end{Verbatim}

From this you can see that you have errors in the file "\verb!main.c!".
When you press <Enter>, Vim displays the file "\verb!main.c!", with the cursor positioned on line 6, the first line with an error.
You did not need to specify the file or the line number, Vim knew where to go by looking in the error messages.

\begin{Verbatim}[samepage=true]
                +---------------------------------------------------+
                |int main()                                         |
                |{                                                  |
                |  int i=3;                                         |
      cursor -> |  do_sub("foo");                                   |
                |  ++i;                                             |
                |  return (0);                                      |
                |}                                                  |
                |}                                                  |
                | ~                                                 |
                |(3 of 12): too many arguments to function 'do_sub' |
                +---------------------------------------------------+
\end{Verbatim}

The following command goes to where the next error occurs:

\begin{Verbatim}[samepage=true]
 :cnext
\end{Verbatim}

Vim jumps to line 10, the last line in the file, where there is an extra '\verb!}!'.

When there is not enough room, Vim will shorten the error message.
To see the whole message use:

\begin{Verbatim}[samepage=true]
 :cc
\end{Verbatim}

You can get an overview of all the error messages with the "\verb!:clist!" command.
The output looks like this:

\begin{Verbatim}[samepage=true]
 :clist
    3 main.c: 6:too many arguments to function 'do_sub' 
    5 main.c: 10:parse error before '}' 
\end{Verbatim}

Only the lines where Vim recognized a file name and line number are listed here.
It assumes those are the interesting lines and the rest is just boring messages.
However, sometimes unrecognized lines do contain something you want to see.
Output from the linker, for example, about an undefined function.
To see all the messages add a "\verb!!!" to the command:

\begin{Verbatim}[samepage=true]
 :clist!
    1 gcc -g -Wall -o prog main.c sub.c 
    2 main.c: In function 'main': 
    3 main.c:6: too many arguments to function 'do_sub' 
    4 main.c: At top level: 
    5 main.c:10: parse error before '}' 
    6 make: *** [prog] Error 1 
\end{Verbatim}

Vim will highlight the current error.
To go back to the previous error, use:

\begin{Verbatim}[samepage=true]
 :cprevious
\end{Verbatim}

Other commands to move around in the error list:

\begin{center} \begin{tabular}{c c}
				\verb!:cfirst! & to first error \\
				\verb!:clast! & to last error \\
				\verb!:cc 3! & to error nr 3 \\
\end{tabular} \end{center}

\subsubsection{Using another compiler}
The name of the program to run when the "\verb!:make!" command is executed is defined by the 'makeprg' option.
Usually this is set to "\verb!make!", but Visual C++ users should set this to "\verb!nmake!" by executing the following command:

\begin{Verbatim}[samepage=true]
 :set makeprg=nmake
\end{Verbatim}

You can also include arguments in this option.
Special characters need to be escaped with a backslash.
Example:

\begin{Verbatim}[samepage=true]
 :set makeprg=nmake\ -f\ project.mak
\end{Verbatim}

You can include special Vim keywords in the command specification.
The \verb!%! character expands to the name of the current file.
So if you execute the command:

\begin{Verbatim}[samepage=true]
 :set makeprg=make\ %
\end{Verbatim}

When you are editing main.c, then "\verb!:make!" executes the following command:

\begin{Verbatim}[samepage=true]
 make main.c
\end{Verbatim}

This is not too useful, so you will refine the command a little and use the \verb!:r! (root) modifier:

\begin{Verbatim}[samepage=true]
 :set makeprg=make\ %:r.o
\end{Verbatim}

Now the command executed is as follows:

\begin{Verbatim}[samepage=true]
 make main.o
\end{Verbatim}

More about these modifiers here: |\verb!:h filename-modifiers!|.
\subsubsection{Old error lists}
Suppose you "\verb!:make!" a program.
There is a warning message in one file and an error message in another.
You fix the error and use "\verb!:make!" again to check if it was really fixed.
Now you want to look at the warning message.
It doesn't show up in the last error list, since the file with the warning wasn't compiled again.
You can go back to the previous error list with:

\begin{Verbatim}[samepage=true]
 :colder
\end{Verbatim}

Then use "\verb!:clist!" and "\verb!:cc {nr}!" to jump to the place with the warning.

To go forward to the next error list:

\begin{Verbatim}[samepage=true]
 :cnewer
\end{Verbatim}

Vim remembers ten error lists.
\subsubsection{Switching compilers}
You have to tell Vim what format the error messages are that your compiler produces.
This is done with the 'errorformat' option.
The syntax of this option is quite complicated and it can be made to fit almost any compiler.
You can find the explanation here: |\verb!:h errorformat!|.

You might be using various different compilers.
Setting the 'makeprg' option, and especially the 'errorformat' each time is not easy.
Vim offers a simple method for this.
For example, to switch to using the Microsoft Visual C++ compiler:

\begin{Verbatim}[samepage=true]
 :compiler msvc
\end{Verbatim}

This will find the Vim script for the "\verb!msvc!" compiler and set the appropriate options.

You can write your own compiler files.  See \hyperref[write-compiler-plugin]{|\texttt{write-compiler-plugin}|}.

\subsubsection{Output redirection}
The "\verb!:make!" command redirects the output of the executed program to an error file.
How this works depends on various things, such as the 'shell'.
If your "\verb!:make!" command doesn't capture the output, check the 'makeef' and 'shellpipe' options.
The 'shellquote' and 'shellxquote' options might also matter.

In case you can't get "\verb!:make!" to redirect the file for you, an alternative is to compile the program in another window and redirect the output into a file.
Then have Vim read this file with:

\begin{Verbatim}[samepage=true]
 :cfile {filename}
\end{Verbatim}

Jumping to errors will work like with the "\verb!:make!" command.
\subsection{Indenting C style text}
A program is much easier to understand when the lines have been properly indented.
Vim offers various ways to make this less work.
For C or C style programs like Java or C++, set the 'cindent' option.
Vim knows a lot about C programs and will try very hard to automatically set the indent for you.
Set the 'shiftwidth' option to the amount of spaces you want for a deeper level.
Four spaces will work fine.
One "\verb!:set!" command will do it:

\begin{Verbatim}[samepage=true]
 :set cindent shiftwidth=4
\end{Verbatim}

With this option enabled, when you type something such as "\verb!if (x)!", the next line will automatically be indented an additional level.

\begin{Verbatim}[samepage=true]
                             if (flag)
    Automatic indent   --->      do_the_work();
    Automatic unindent <--   if (other_flag) {
    Automatic indent   --->      do_file();
    keep indent                  do_some_more();
    Automatic unindent <--   }
\end{Verbatim}

When you type something in curly braces (\verb!{}!), the text will be indented at the start and unindented at the end.
The unindenting will happen after typing the '\verb!}!', since Vim can't guess what you are going to type.

One side effect of automatic indentation is that it helps you catch errors in your code early.
When you type a \verb!}! to finish a function, only to find that the automatic indentation gives it more indent than what you expected, there is probably a \verb!}! missing.
Use the "\verb!%!" command to find out which \verb!{! matches the \verb!}! you typed.

A missing \verb!)! and \verb!;! also cause extra indent.
Thus if you get more white space than you would expect, check the preceding lines.

When you have code that is badly formatted, or you inserted and deleted lines, you need to re-indent the lines.
The "\verb!=!" operator does this.
The simplest form is:

\begin{Verbatim}[samepage=true]
 ==
\end{Verbatim}

This indents the current line.
Like with all operators, there are three ways to use it.
In Visual mode "\verb!=!" indents the selected lines.
A useful text object is "\verb!a{!".
This selects the current \verb!{}! block.
Thus, to re-indent the code block the cursor is in:

\begin{Verbatim}[samepage=true]
 =a{
\end{Verbatim}

I you have really badly indented code, you can re-indent the whole file with:

\begin{Verbatim}[samepage=true]
 gg=G
\end{Verbatim}

However, don't do this in files that have been carefully indented manually.
The automatic indenting does a good job, but in some situations you might want to overrule it.
\subsubsection{Setting indent style}
Different people have different styles of indentation.
By default Vim does a pretty good job of indenting in a way that 90% of programmers do.
There are different styles, however; so if you want to, you can customize the indentation style with the 'cinoptions' option.

By default 'cinoptions' is empty and Vim uses the default style.
You can add various items where you want something different.
For example, to make curly braces be placed like this:

\begin{Verbatim}[samepage=true]
    if (flag) 
      { 
        i = 8; 
        j = 0; 
      } 
\end{Verbatim}

Use this command:

\begin{Verbatim}[samepage=true]
 :set cinoptions+={2
\end{Verbatim}

There are many of these items.  See |\verb!:h cinoptions-values!|.
\subsection{Automatic indenting}
You don't want to switch on the 'cindent' option manually every time you edit a C file.
This is how you make it work automatically:

\begin{Verbatim}[samepage=true]
 :filetype indent on
\end{Verbatim}

Actually, this does a lot more than switching on 'cindent' for C files.
First of all, it enables detecting the type of a file.
That's the same as what is used for syntax highlighting.

When the filetype is known, Vim will search for an indent file for this type of file.
The Vim distribution includes a number of these for various programming languages.
This indent file will then prepare for automatic indenting specifically for this file.

If you don't like the automatic indenting, you can switch it off again:

\begin{Verbatim}[samepage=true]
 :filetype indent off
\end{Verbatim}

If you don't like the indenting for one specific type of file, this is how you avoid it.
Create a file with just this one line:

\begin{Verbatim}[samepage=true]
 :let b:did_indent = 1
\end{Verbatim}

Now you need to write this in a file with a specific name:

\begin{Verbatim}[samepage=true]
    {directory}/indent/{filetype}.vim
\end{Verbatim}

The {filetype} is the name of the file type, such as "\verb!cpp!" or "\verb!java!".
You can see the exact name that Vim detected with this command:

\begin{Verbatim}[samepage=true]
 :set filetype
\end{Verbatim}

In this file the output is:

\begin{Verbatim}[samepage=true]
    filetype=help 
\end{Verbatim}

Thus you would use "\verb!help!" for \verb!{filetype}!.

For the {directory} part you need to use your runtime directory.
Look at the output of this command:

\begin{Verbatim}[samepage=true]
 set runtimepath
\end{Verbatim}

Now use the first item, the name before the first comma.
Thus if the output looks like this:

\begin{Verbatim}[samepage=true]
    runtimepath=~/.vim,/usr/local/share/vim/vim60/runtime,~/.vim/after 
\end{Verbatim}

You use "\verb!~/.vim!" for \verb!{directory}!.
Then the resulting file name is:

\begin{Verbatim}[samepage=true]
    ~/.vim/indent/help.vim 
\end{Verbatim}

Instead of switching the indenting off, you could write your own indent file.
How to do that is explained here: |\verb!:h indent-expression!|.
\subsection{Other indenting}
The most simple form of automatic indenting is with the 'autoindent' option.
It uses the indent from the previous line.
A bit smarter is the 'smartindent' option.
This is useful for languages where no indent file is available.
'smartindent' is not as smart as 'cindent', but smarter than 'autoindent'.

With 'smartindent' set, an extra level of indentation is added for each { and removed for each }.
An extra level of indentation will also be added for any of the words in the 'cinwords' option.
Lines that begin with \verb!#! are treated specially: all indentation is removed.
This is done so that preprocessor directives will all start in column 1.
The indentation is restored for the next line.

\subsubsection{Correcting indents}
When you are using 'autoindent' or 'smartindent' to get the indent of the previous line, there will be many times when you need to add or remove one 'shiftwidth' worth of indent.
A quick way to do this is using the CTRL-D and CTRL-T commands in Insert mode.

For example, you are typing a shell script that is supposed to look like this:

\begin{Verbatim}[samepage=true]
    if test -n a; then 
       echo a 
       echo "-------" 
    fi 
\end{Verbatim}

Start off by setting these options:

\begin{Verbatim}[samepage=true]
 :set autoindent shiftwidth=3
\end{Verbatim}

You start by typing the first line, <Enter> and the start of the second line:

\begin{Verbatim}[samepage=true]
    if test -n a; then 
    echo 
\end{Verbatim}

Now you see that you need an extra indent.
Type CTRL-T.
The result:

\begin{Verbatim}[samepage=true]
    if test -n a; then 
         echo 
\end{Verbatim}

The CTRL-T command, in Insert mode, adds one 'shiftwidth' to the indent, no matter where in the line you are.

You continue typing the second line, <Enter> and the third line.
This time the indent is OK.
Then <Enter> and the last line.
Now you have this:

\begin{Verbatim}[samepage=true]
    if test -n a; then 
         echo a 
         echo "-------" 
       fi 
\end{Verbatim}

To remove the superfluous indent in the last line press CTRL-D.
This deletes one 'shiftwidth' worth of indent, no matter where you are in the line.

When you are in Normal mode, you can use the "\verb!>>!" and "\verb!<<!" commands to shift lines.
"\verb!>!" and "\verb!<!" are operators, thus you have the usual three ways to specify the lines you want to indent.
A useful combination is:

\begin{Verbatim}[samepage=true]
 >i{
\end{Verbatim}

This adds one indent to the current block of lines, inside {}.
The { and } lines themselves are left unmodified.
"\verb!>a{!" includes them.
In this example the cursor is on "\verb!printf!":

% The minipage solution isn't the most elegant but works well to integrate the verbatim inside the cells
% cf https://stackoverflow.com/questions/3220121/verbatim-environment-inside-latex-cell
\begin{center} \begin{tabular}{|c|c|c|}
				\hline
				original text & after "\verb!>i{!" & after "\verb!>a{!" \\ 
				\begin{minipage}{4cm}
				\begin{verbatim}
if (flag)       
{               
printf("yes");  
flag = 0;       
}
				\end{verbatim}
				\end{minipage}
& 
				\begin{minipage}{4cm}
				\begin{verbatim}
if (flag)         
{                 
  printf("yes");
  flag = 0; 
}                 
				\end{verbatim}
				\end{minipage}
&
				\begin{minipage}{4cm}
				\begin{verbatim}
if (flag) 
  { 
  printf("yes"); 
  flag = 0;  
  } 
				\end{verbatim}
				\end{minipage} \\
				\hline
\end{tabular} \end{center}
\subsection{Tabs and spaces}
'tabstop' is set to eight by default.
Although you can change it, you quickly run into trouble later.
Other programs won't know what tabstop value you used.
They probably use the default value of eight, and your text suddenly looks very different.
Also, most printers use a fixed tabstop value of eight.
Thus it's best to keep 'tabstop' alone.
(If you edit a file which was written with a different tabstop setting, see |\hyperref[Indents and tabs]{\texttt{Indents and tabs}}| for how to fix that.)

For indenting lines in a program, using a multiple of eight spaces makes you quickly run into the right border of the window.
Using a single space doesn't provide enough visual difference.
Many people prefer to use four spaces, a good compromise.

Since a <Tab> is eight spaces and you want to use an indent of four spaces, you can't use a <Tab> character to make your indent.
There are two ways to handle this:

\begin{enumerate}
				\item Use a mix of <Tab> and space characters.
								Since a <Tab> takes the place of eight spaces, you have fewer characters in your file.
								Inserting a <Tab> is quicker than eight spaces.
								Backspacing works faster as well.
				\item Use spaces only.
								This avoids the trouble with programs that use a different tabstop value.
\end{enumerate}

Fortunately, Vim supports both methods quite well.
\subsubsection{Spaces and tabs}
If you are using a combination of tabs and spaces, you just edit normally.
The Vim defaults do a fine job of handling things.

You can make life a little easier by setting the 'softtabstop' option.
This option tells Vim to make the <Tab> key look and feel as if tabs were set at the value of 'softtabstop', but actually use a combination of tabs and spaces.

After you execute the following command, every time you press the <Tab> key the cursor moves to the next 4-column boundary:

\begin{Verbatim}[samepage=true]
 :set softtabstop=4
\end{Verbatim}

When you start in the first column and press <Tab>, you get 4 spaces inserted in your text.
The second time, Vim takes out the 4 spaces and puts in a <Tab> (thus taking you to column 8).
Thus Vim uses as many <Tab>s as possible, and then fills up with spaces.

When backspacing it works the other way around.
A <BS> will always delete the amount specified with 'softtabstop'.
Then <Tab>s are used as many as possible and spaces to fill the gap.

The following shows what happens pressing <Tab> a few times, and then using <BS>.
A "\verb!.!" stands for a space and "\verb!------->!" for a <Tab>.

\begin{center} \begin{tabular}{l l}
type & result \\ 
\verb!<Tab>! & \verb!....! \\
\verb!<Tab><Tab>! & \verb!------->! \\
\verb!<Tab><Tab><Tab>! & \verb!------->....! \\
\verb!<Tab><Tab><Tab><BS>! & \verb!------->! \\
\verb!<Tab><Tab><Tab><BS><BS>! & \verb!....! \\		
\end{tabular} \end{center}

An alternative is to use the 'smarttab' option.
When it's set, Vim uses 'shiftwidth' for a <Tab> typed in the indent of a line, and a real <Tab> when typed after the first non-blank character.
However, <BS> doesn't work like with 'softtabstop'.

\subsubsection{Just spaces}
If you want absolutely no tabs in your file, you can set the 'expandtab' option:

\begin{Verbatim}[samepage=true]
 :set expandtab
\end{Verbatim}

When this option is set, the <Tab> key inserts a series of spaces.
Thus you get the same amount of white space as if a <Tab> character was inserted, but there isn't a real <Tab> character in your file.

The backspace key will delete each space by itself.
Thus after typing one <Tab> you have to press the <BS> key up to eight times to undo it.
If you are in the indent, pressing CTRL-D will be a lot quicker.

\subsubsection{Changing tabs in spaces (and back)}
Setting 'expandtab' does not affect any existing tabs.
In other words, any tabs in the document remain tabs.
If you want to convert tabs to spaces, use the "\verb!:retab!" command.
Use these commands:

\begin{Verbatim}[samepage=true]
 :set expandtab
 :%retab
\end{Verbatim}

Now Vim will have changed all indents to use spaces instead of tabs.
However, all tabs that come after a non-blank character are kept.
If you want these to be converted as well, add a \verb,!,:

\begin{Verbatim}[samepage=true]
 :%retab!
\end{Verbatim}

This is a little bit dangerous, because it can also change tabs inside a string.
To check if these exist, you could use this:

\begin{Verbatim}[samepage=true]
 /"[^"\t]*\t[^"]*"
\end{Verbatim}

It's recommended not to use hard tabs inside a string.
Replace them with "\verb!\t!" to avoid trouble.

The other way around works just as well:

\begin{Verbatim}[samepage=true]
 :set noexpandtab
 :%retab!
\end{Verbatim}
\subsection{Formatting comments}
One of the great things about Vim is that it understands comments.
You can ask Vim to format a comment and it will do the right thing.

Suppose, for example, that you have the following comment:

\begin{Verbatim}[samepage=true]
    /* 
     * This is a test 
     * of the text formatting. 
     */ 
\end{Verbatim}

You then ask Vim to format it by positioning the cursor at the start of the comment and type:

\begin{Verbatim}[samepage=true]
 gq]/
\end{Verbatim}

"\verb!gq!" is the operator to format text.
"\verb!]/!" is the motion that takes you to the end of a comment.
The result is:

\begin{Verbatim}[samepage=true]
    /* 
     * This is a test of the text formatting. 
     */ 
\end{Verbatim}

Notice that Vim properly handled the beginning of each line.
An alternative is to select the text that is to be formatted in Visual mode and type "\verb!gq!".

To add a new line to the comment, position the cursor on the middle line and press "\verb!o!".
The result looks like this:

\begin{Verbatim}[samepage=true]
    /* 
     * This is a test of the text formatting. 
     * 
     */ 
\end{Verbatim}

Vim has automatically inserted a star and a space for you.
Now you can type the comment text.
When it gets longer than 'textwidth', Vim will break the line.
Again, the star is inserted automatically:

\begin{Verbatim}[samepage=true]
    /* 
     * This is a test of the text formatting. 
     * Typing a lot of text here will make Vim 
     * break 
     */ 
\end{Verbatim}

For this to work some flags must be present in 'formatoptions':

\begin{center} \begin{tabular}{c l}
\verb!r! & insert the star when typing <Enter> in Insert mode \\
\verb!o! & insert the star when using "\verb!o!" or "\verb!O!" in Normal mode \\
\verb!c! & break comment text according to 'textwidth' \\
\end{tabular} \end{center}

See |\verb!:h fo-table!| for more flags.

\subsubsection{Defining a comment}
The 'comments' option defines what a comment looks like.
Vim distinguishes between a single-line comment and a comment that has a different start, end and middle part.

Many single-line comments start with a specific character.
In C++ \verb!//! is used, in Makefiles \verb!#!, in Vim scripts \verb!"!.
For example, to make Vim understand C++ comments:

\begin{Verbatim}[samepage=true]
 :set comments=://
\end{Verbatim}

The colon separates the flags of an item from the text by which the comment is recognized.
The general form of an item in 'comments' is:

\begin{Verbatim}[samepage=true]
    {flags}:{text}
\end{Verbatim}

The \verb!{flags}! part can be empty, as in this case.

Several of these items can be concatenated, separated by commas.
This allows recognizing different types of comments at the same time.
For example, let's edit an e-mail message.
When replying, the text that others wrote is preceded with "\verb!>!" and "\verb!!!" characters.
This command would work:

\begin{Verbatim}[samepage=true]
 :set comments=n:>,n:!
\end{Verbatim}

There are two items, one for comments starting with "\verb!>!" and one for comments that start with "\verb!!!".
Both use the flag "\verb!n!".
This means that these comments nest.
Thus a line starting with "\verb!>!" may have another comment after the "\verb!>!".
This allows formatting a message like this:

\begin{Verbatim}[samepage=true]
    > ! Did you see that site? 
    > ! It looks really great. 
    > I don't like it.  The 
    > colors are terrible. 
    What is the URL of that 
    site? 
\end{Verbatim}

Try setting 'textwidth' to a different value, e.g., 80, and format the text by Visually selecting it and typing "\verb!gq!".
The result is:

\begin{Verbatim}[samepage=true]
    > ! Did you see that site?  It looks really great. 
    > I don't like it.  The colors are terrible. 
    What is the URL of that site? 
\end{Verbatim}

You will notice that Vim did not move text from one type of comment to another.
The "\verb!I!" in the second line would have fit at the end of the first line, but since that line starts with "\verb!> !!" and the second line with "\verb!>!", Vim knows that this is a different kind of comment.

\subsubsection{A three part comment}
A C comment starts with "\verb!/*!", has "\verb!*!" in the middle and "\verb!*/!" at the end.
The entry in 'comments' for this looks like this:

\begin{Verbatim}[samepage=true]
 :set comments=s1:/*,mb:*,ex:*/
\end{Verbatim}

The start is defined with "\verb!s1:/*!".
The "\verb!s!" indicates the start of a three-piece comment.
The colon separates the flags from the text by which the comment is recognized: "\verb!/*!".
There is one flag: "\verb!1!".
This tells Vim that the middle part has an offset of one space.

The middle part "\verb!mb:*!" starts with "\verb!m!", which indicates it is a middle part.
The "\verb!b!" flag means that a blank must follow the text.
Otherwise Vim would consider text like "\verb!*pointer!" also to be the middle of a comment.

The end part "\verb!ex:*/!" has the "\verb!e!" for identification.
The "\verb!x!" flag has a special meaning.
It means that after Vim automatically inserted a star, typing / will remove the extra space.

For more details see |\verb!:h format-comments!|.
\clearpage
