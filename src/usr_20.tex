\section{Typing command-line commands quickly}
Vim has a few generic features that makes it easier to enter commands.  Colon
commands can be abbreviated, edited and repeated.  Completion is available for
nearly everything.
\localtableofcontentswithrelativedepth{+1}
\subsection{Command line editing}
When you use a colon (\verb!:!) command or search for a string with \verb!/! or \verb!?!, Vim puts the cursor on the bottom of the screen.
There you type the command or search pattern.
This is called the Command line.
Also when it's used for entering a search command.

The most obvious way to edit the command you type is by pressing the <BS> key.
This erases the character before the cursor.
To erase another character, typed earlier, first move the cursor with the cursor keys.

For example, you have typed this:

\begin{Verbatim}[samepage=true]
 :s/col/pig/
\end{Verbatim}

Before you hit <Enter>, you notice that "col" should be "cow".
To correct this, you type <Left> five times.
The cursor is now just after "col".
Type <BS> and "\verb!w!" to correct:

\begin{Verbatim}[samepage=true]
 :s/cow/pig/
\end{Verbatim}

Now you can press <Enter> directly.
You don't have to move the cursor to the end of the line before executing the command.

The most often used keys to move around in the command line:

\begin{center} \begin{longtable}{c l}
				<Left> & one character left \\
				<Right> & one character right\\
				<S-Left> or <C-Left> & one word left\\
				<S-Right> or <C-Right> & one word right\\
				CTRL-B or <Home> & to begin of command line\\
				CTRL-E or <End> & to end of command line\\
\end{longtable} \end{center}

\emph{Note}:
<S-Left> (cursor left key with Shift key pressed) and <C-Left> (cursor left key with Control pressed) will not work on all keyboards.
Same for the other Shift and Control combinations.

You can also use the mouse to move the cursor.
\subsubsection{Deleting}
As mentioned, <BS> deletes the character before the cursor.  To delete a whole
word use CTRL-W.

\begin{Verbatim}[samepage=true]
    /the fine pig 

             CTRL-W

    /the fine 
\end{Verbatim}

CTRL-U removes all text, thus allows you to start all over again.
\subsubsection{Overstrike}
The <Insert> key toggles between inserting characters and replacing the existing ones.
Start with this text:

\begin{Verbatim}[samepage=true]
    /the fine pig 
\end{Verbatim}

Move the cursor to the start of "fine" with <S-Left> twice (or <Left> eight times, if <S-Left> doesn't work).
Now press <Insert> to switch to overstrike and type "great":

\begin{Verbatim}[samepage=true]
    /the greatpig 
\end{Verbatim}

Oops, we lost the space.
Now, don't use <BS>, because it would delete the "t" (this is different from Replace mode).
Instead, press <Insert> to switch from overstrike to inserting, and type the space:

\begin{Verbatim}[samepage=true]
    /the great pig 
\end{Verbatim}

\subsubsection{Cancelling}
You thought of executing a \verb!:! or \verb!/! command, but changed your mind.
To get rid of what you already typed, without executing it, press CTRL-C or <Esc>.

\emph{Note}:
<Esc> is the universal "get out" key.
Unfortunately, in the good old Vi pressing <Esc> in a command line executed the command!
Since that might be considered to be a bug, Vim uses <Esc> to cancel the command.
But with the 'cpoptions' option it can be made Vi compatible.
And when using a mapping (which might be written for Vi) <Esc> also works Vi compatible.
Therefore, using CTRL-C is a method that always works.

If you are at the start of the command line, pressing <BS> will cancel the command.
It's like deleting the "\verb!:!" or "\verb!/!" that the line starts with.
\subsection{Command line abbreviations}
Some of the "\verb!:!" commands are really long.
We already mentioned that "\verb!:substitute!" can be abbreviated to "\verb!:s!".
This is a generic mechanism, all "\verb!:!" commands can be abbreviated.

How short can a command get?  There are 26 letters, and many more commands.
For example, "\verb!:set!" also starts with "\verb!:s!", but "\verb!:s!" doesn't start a "\verb!:set!" command.
Instead "\verb!:set!" can be abbreviated to "\verb!:se!".

When the shorter form of a command could be used for two commands, it stands for only one of them.
There is no logic behind which one, you have to learn them.
In the help files the shortest form that works is mentioned.
For example:

\begin{Verbatim}[samepage=true]
 :s[ubstitute]
\end{Verbatim}

This means that the shortest form of "\verb!:substitute!" is "\verb!:s!".
The following characters are optional.
Thus "\verb!:su!" and "\verb!:sub!" also work.

In the user manual we will either use the full name of command, or a short version that is still readable.
For example, "\verb!:function!" can be abbreviated to "\verb!:fu!".
But since most people don't understand what that stands for, we will use "\verb!:fun!".
(Vim doesn't have a ":funny" command, otherwise ":fun" would be confusing too.)

It is recommended that in Vim scripts you write the full command name.
That makes it easier to read back when you make later changes.
Except for some often used commands like "\verb!:w!" ("\verb!:write!") and "\verb!:r!" ("\verb!:read!").

A particularly confusing one is "\verb!:end!", which could stand for "\verb!:endif!", "\verb!:endwhile!" or "\verb!:endfunction!".
Therefore, always use the full name.
\subsubsection{Short option names}
In the user manual the long version of the option names is used.
Many options also have a short name.
Unlike "\verb!:!" commands, there is only one short name that works.
For example, the short name of 'autoindent' is 'ai'.
Thus these two commands do the same thing:

\begin{Verbatim}[samepage=true]
 :set autoindent
 :set ai
\end{Verbatim}

You can find the full list of long and short names here: |\verb!:h option-list!|.
\subsection{Command line completion}
This is one of those Vim features that, by itself, is a reason to switch from Vi to Vim.
Once you have used this, you can't do without.

Suppose you have a directory that contains these files:

\begin{Verbatim}[samepage=true]
    info.txt
    intro.txt
    bodyofthepaper.txt
\end{Verbatim}

To edit the last one, you use the command:

\begin{Verbatim}[samepage=true]
 :edit bodyofthepaper.txt
\end{Verbatim}

It's easy to type this wrong.
A much quicker way is:

\begin{Verbatim}[samepage=true]
 :edit b<Tab>
\end{Verbatim}

Which will result in the same command.
What happened?
The <Tab> key does completion of the word before the cursor.
In this case "b".
Vim looks in the directory and finds only one file that starts with a "b".
That must be the one you are looking for, thus Vim completes the file name for you.

Now type:

\begin{Verbatim}[samepage=true]
 :edit i<Tab>
\end{Verbatim}

Vim will beep, and give you:

\begin{Verbatim}[samepage=true]
 :edit info.txt
\end{Verbatim}

The beep means that Vim has found more than one match.
It then uses the first match it found (alphabetically).
If you press <Tab> again, you get:

\begin{Verbatim}[samepage=true]
 :edit intro.txt
\end{Verbatim}

Thus, if the first <Tab> doesn't give you the file you were looking for, press it again.
If there are more matches, you will see them all, one at a time.

If you press <Tab> on the last matching entry, you will go back to what you first typed:

\begin{Verbatim}[samepage=true]
 :edit i
\end{Verbatim}

Then it starts all over again.
Thus Vim cycles through the list of matches.
Use CTRL-P to go through the list in the other direction:

\begin{Verbatim}[samepage=true]
          <------------------- <Tab> -------------------------+
                                                              |
             <Tab> -->                   <Tab> -->
    :edit i              :edit info.txt                :edit intro.txt
             <-- CTRL-P                  <-- CTRL-P
       |
       +---------------------- CTRL-P ------------------------>
\end{Verbatim}
\subsubsection{Context}
When you type "\verb!:set i!" instead of "\verb!:edit i!" and press <Tab> you get:

\begin{Verbatim}[samepage=true]
 :set icon
\end{Verbatim}

Hey, why didn't you get "\verb!:set info.txt!"?
That's because Vim has context sensitive completion.
The kind of words Vim will look for depends on the command before it.
Vim knows that you cannot use a file name just after a "\verb!:set!" command, but you can use an option name.

Again, if you repeat typing the <Tab>, Vim will cycle through all matches.
There are quite a few, it's better to type more characters first:

\begin{Verbatim}[samepage=true]
 :set isk<Tab>
\end{Verbatim}

Gives:

\begin{Verbatim}[samepage=true]
 :set iskeyword
\end{Verbatim}

Now type "\verb!=!" and press <Tab>:

\begin{Verbatim}[samepage=true]
 :set iskeyword=@,48-57,_,192-255
\end{Verbatim}

What happens here is that Vim inserts the old value of the option.
Now you can edit it.

What is completed with <Tab> is what Vim expects in that place.
Just try it out to see how it works.
In some situations you will not get what you want.
That's either because Vim doesn't know what you want, or because completion was not implemented for that situation.
In that case you will get a <Tab> inserted (displayed as \textasciicircum I).
\subsubsection{List matches}
When there are many matches, you would like to see an overview.
Do this by pressing CTRL-D.
For example, pressing CTRL-D after:

\begin{Verbatim}[samepage=true]
 :set is
\end{Verbatim}

results in:

\begin{Verbatim}[samepage=true]
 :set is
 incsearch  isfname    isident    iskeyword  isprint
 :set is
\end{Verbatim}

Vim lists the matches and then comes back with the text you typed.
You can now check the list for the item you wanted.
If it isn't there, you can use <BS> to correct the word.
If there are many matches, type a few more characters before pressing <Tab> to complete the rest.

If you have watched carefully, you will have noticed that "\verb!incsearch!" doesn't start with "is".
In this case "is" stands for the short name of "\verb!incsearch!".
(Many options have a short and a long name.)  Vim is clever enough to know that you might have wanted to expand the short name of the option into the long name.
\subsubsection{There is more}
The CTRL-L command completes the word to the longest unambiguous string.
If you type "\verb!:edit i!" and there are files "\verb!info.txt!" and "\verb!info_backup.txt!" you will get "\verb!:edit info!".

The 'wildmode' option can be used to change the way completion works.
The 'wildmenu' option can be used to get a menu-like list of matches.
Use the 'suffixes' option to specify files that are less important and appear at the end of the list of files.
The 'wildignore' option specifies files that are not listed at all.

More about all of this here: |\verb!:h cmdline-completion!|
\subsection{Command line history}
In chapter 3 we briefly mentioned the history.
The basics are that you can use the <Up> key to recall an older command line.
<Down> then takes you back to newer commands.

There are actually four histories.
The ones we will mention here are for "\verb!:!" commands and for "\verb!/!" and "\verb!?!" search commands.
The "/" and "?" commands share the same history, because they are both search commands.
The two other histories are for expressions and input lines for the \verb!input()! function.
|\verb!:h cmdline-history!|

Suppose you have done a "\verb!:set!" command, typed ten more colon commands and then want to repeat that "\verb!:set!" command again.
You could press "\verb!:!" and then ten times <Up>.
There is a quicker way:

\begin{Verbatim}[samepage=true]
 :se<Up>
\end{Verbatim}

Vim will now go back to the previous command that started with "\verb!se!".
You have a good chance that this is the "\verb!:set!" command you were looking for.
At least you should not have to press <Up> very often (unless "\verb!:set!" commands is all you have done).

The <Up> key will use the text typed so far and compare it with the lines in the history.
Only matching lines will be used.

If you do not find the line you were looking for, use <Down> to go back to what you typed and correct that.
Or use CTRL-U to start all over again.

To see all the lines in the history:

\begin{Verbatim}[samepage=true]
 :history
\end{Verbatim}

That's the history of "\verb!:!" commands.
The search history is displayed with this command:

\begin{Verbatim}[samepage=true]
 :history /
\end{Verbatim}

CTRL-P will work like <Up>, except that it doesn't matter what you already typed.
Similarly for CTRL-N and <Down>.
CTRL-P stands for previous, CTRL-N for next.
\subsection{Command line window}
Typing the text in the command line works different from typing text in Insert mode.
It doesn't allow many commands to change the text.
For most commands that's OK, but sometimes you have to type a complicated command.
That's where the command line window is useful.

Open the command line window with this command:

\begin{Verbatim}[samepage=true]
 q:
\end{Verbatim}

Vim now opens a (small) window at the bottom.
It contains the command line history, and an empty line at the end:

\begin{Verbatim}[samepage=true]
    +-------------------------------------+
    |other window                         |
    |~                                    |
    |\verb!:h file.txt=============================!|
    |:e c                                 |
    |:e config.h.in                       |
    |:set path=.,/usr/include,,           |
    |:set iskeyword=@,48-57,_,192-255     |
    |:set is                              |
    |:q                                   |
    |:                                    |
    |\verb!:h command-line=========================!|
    |                                     |
    +-------------------------------------+
\end{Verbatim}

You are now in Normal mode.
You can use the "\verb!hjkl!" keys to move around.
For example, move up with "\verb!5k!" to the "\verb!:e config.h.in!" line.
Type "\verb!$h!" to go to the "\verb!i!" of "\verb!in!" and type "\verb!cwout!".
Now you have changed the line to:

\begin{Verbatim}[samepage=true]
    :e config.h.out 
\end{Verbatim}

Now press <Enter> and this command will be executed.
The command line window will close.

The <Enter> command will execute the line under the cursor.
It doesn't matter whether Vim is in Insert mode or in Normal mode.

Changes in the command line window are lost.
They do not result in the history to be changed.
Except that the command you execute will be added to the end of the history, like with all executed commands.

The command line window is very useful when you want to have overview of the history, lookup a similar command, change it a bit and execute it.
A search command can be used to find something.

In the previous example the "\verb!?config!" search command could have been used to find the previous command that contains "config".
It's a bit strange, because you are using a command line to search in the command line window.
While typing that search command you can't open another command line window, there can be only one.
\clearpage
