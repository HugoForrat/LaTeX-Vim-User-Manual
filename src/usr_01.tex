\section{01. About the manuals}
This chapter introduces the manuals available with Vim.
Read this to know the conditions under which the commands are explained.
\localtableofcontentswithrelativedepth{+1}
\subsection{Two manuals}

The Vim documentation consists of two parts:
\begin{description}
				\item[The User manual]
								Task oriented explanations, from simple to complex.
								Reads from start to end like a book.
				\item[The Reference manual]
								Precise description of how everything in Vim works.
\end{description}
The notation used in these manuals is explained here: \hyperref[notation]{|\texttt{notation}|}

\textit{Note for the \LaTeX edition :}
In the original Vim manual, link are presented inbetween \verb!|! symbol.
This notation is kept here.
But, those links can lead to another part of this manual or to the Vim reference manual.
In the first case, the links work like any hyperlink you can find in a pdf document.
Regarding the links leading to the reference manual, they are prefixed in this "edition" with \verb`:h`.
Typing this command inside of Vim should lead to the correct part of the reference manual.
Some links to the User Manual were represented by the number corresponding to the subsection they were linked to.
They're represented here by the title of the subsection instead.
There is normally here a subsubsection about Jumping Around which have been deleted from the pdf edition because most of the things it explained didn't apply to a pdf file.
However they do apply when reading the reference manual inside of Vim, which most users do.
You can read the subsubsection by typing \verb!:h 01.1! in Vim.

% JUMPING AROUND
% 
% The text contains hyperlinks between the two parts, allowing you to quickly jump between the description of an editing task and a precise explanation of the commands and options used for it.
% Use these two commands:
% 
% 		Press CTRL-] to jump to a subject under the cursor.
% 		Press CTRL-O to jump back (repeat to go further back).
% 
% Many links are in vertical bars, like this: |\verb!:h bars!|.
% The bars themselves may
% be hidden or invisible, see below.
% An option name, like 'number', a command in double quotes like ":write" and any other word can also be used as a link.
% Try it out: Move the cursor to CTRL-] and press CTRL-] on it.
% 
% Other subjects can be found with the ":help" command, see |\verb!:h help.txt!|.
% 
% The bars and stars are usually hidden with the |\verb!:h conceal!| feature.
% They also
% use |\verb!:h hl-Ignore!|, using the same color for the text as the background.
% You can
% make them visible with:
% \begin{verbatim}
%  :set conceallevel=0
%  :hi link HelpBar Normal
%  :hi link HelpStar Normal
% \end{verbatim}

\subsection{Vim installed}
Most of the manuals assume that Vim has been properly installed.
If you didn't do that yet, or if Vim doesn't run properly (e.g., files can't be found or in the GUI the menus do not show up) first read the chapter on installation: |\hyperref[Installing Vim]{\texttt{Installing Vim}}|.

\phantomsection
\label{not-compatible}
The manuals often assume you are using Vim with Vi-compatibility switched off.
For most commands this doesn't matter, but sometimes it is important, e.g., for multi-level undo.
An easy way to make sure you are using a nice setup is to copy the example vimrc file.
By doing this inside Vim you don't have to check out where it is located.
How to do this depends on the system you are using:

Unix:
\begin{verbatim}
 :!cp -i $VIMRUNTIME/vimrc_example.vim ~/.vimrc
\end{verbatim}
MS-DOS, MS-Windows, OS/2:
\begin{verbatim}
 :!copy $VIMRUNTIME/vimrc_example.vim $VIM/_vimrc
\end{verbatim}
Amiga:
\begin{verbatim}
 :!copy $VIMRUNTIME/vimrc_example.vim $VIM/.vimrc
\end{verbatim}

If the file already exists you probably want to keep it.

If you start Vim now, the 'compatible' option should be off.
You can check it with this command:

\begin{verbatim}
 :set compatible?
\end{verbatim}

If it responds with "nocompatible" you are doing well.
If the response is "compatible" you are in trouble.
You will have to find out why the option is still set.
Perhaps the file you wrote above is not found.
Use this command to find out:

\begin{verbatim}
 :scriptnames
\end{verbatim}

If your file is not in the list, check its location and name.
If it is in the list, there must be some other place where the 'compatible' option is switched back on.

For more info see |vimrc| and |\verb!:h compatible-default!|.

\textit{Note:}\newline
This manual is about using Vim in the normal way.
There is an alternative called "evim" (easy Vim).
This is still Vim, but used in a way that resembles a click-and-type editor like Notepad.
It always stays in Insert mode, thus it feels very different.
It is not explained in the user manual, since it should be mostly self explanatory.
See |\verb!:h evim-keys!| for details.

\subsection{Using the Vim tutor}
\label{tutor}
\label{vimtutor}

Instead of reading the text (boring!) you can use the vimtutor to learn your first Vim commands.
This is a 30 minute tutorial that teaches the most basic Vim functionality hands-on.

On Unix, if Vim has been properly installed, you can start it from the shell:

\begin{verbatim}
 vimtutor
\end{verbatim}

On MS-Windows you can find it in the Program/Vim menu.
Or execute \verb!vimtutor.bat! in the \$VIMRUNTIME directory.

This will make a copy of the tutor file, so that you can edit it without the risk of damaging the original.
There are a few translated versions of the tutor.
To find out if yours is available, use the two-letter language code.
For French:

\begin{verbatim}
 vimtutor fr
\end{verbatim}

On Unix, if you prefer using the GUI version of Vim, use "gvimtutor" or
"vimtutor -g" instead of "vimtutor".

For OpenVMS, if Vim has been properly installed, you can start vimtutor from a
VMS prompt with:

\begin{verbatim}
 @VIM:vimtutor
\end{verbatim}

Optionally add the two-letter language code as above.

On other systems, you have to do a little work:
\begin{enumerate}
				\item Copy the tutor file. You can do this with Vim (it knows where to find it):
								\begin{verbatim}
 vim -u NONE -c 'e $VIMRUNTIME/tutor/tutor' -c 'w! TUTORCOPY' -c 'q'
								\end{verbatim}
								This will write the file "TUTORCOPY" in the current directory.
								To use a
								translated version of the tutor, append the two-letter language code to the
								filename. For French:
								\begin{verbatim}
 vim -u NONE -c 'e $VIMRUNTIME/tutor/tutor.fr' -c 'w! TUTORCOPY' -c 'q'
								\end{verbatim}

				\item Edit the copied file with Vim:
								\begin{verbatim}
 vim -u NONE -c "set nocp" TUTORCOPY
								\end{verbatim}
								The extra arguments make sure Vim is started in a good mood.

				\item Delete the copied file when you are finished with it:
								\begin{verbatim}
 del TUTORCOPY
								\end{verbatim}
\end{enumerate}
\subsection{Copyright}
\label{manual-copyright}

The Vim user manual and reference manual are Copyright (c) 1988-2003 by Bram Moolenaar.
This material may be distributed only subject to the terms and conditions set forth in the Open Publication License, v1.0 or later.
The latest version is presently available at: \url{http://www.opencontent.org/openpub/}.

People who contribute to the manuals must agree with the above copyright notice.

\phantomsection
\label{frombook}
Parts of the user manual come from the book \textit{Vi IMproved - Vim} by Steve Oualline (published by New Riders Publishing, ISBN: 0735710015).
The Open Publication License applies to this book.
Only selected parts are included and these have been modified (e.g., by removing the pictures, updating the text for Vim 6.0 and later, fixing mistakes).
The omission of the \hyperref[frombook]{|\texttt{frombook}|} tag does not mean that the text does not come from the book.

Many thanks to Steve Oualline and New Riders for creating this book and publishing it under the OPL!
It has been a great help while writing the user manual.
Not only by providing literal text, but also by setting the tone and style.

If you make money through selling the manuals, you are strongly encouraged to donate part of the profit to help AIDS victims in Uganda.
See |\verb!:h iccf!|.
\clearpage
