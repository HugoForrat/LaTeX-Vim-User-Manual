\section{00. Introduction to Vim}
\label{ref}
\label{reference}
\localtableofcontentswithrelativedepth{+1}
\subsection{Introduction}
\label{intro}
Vim stands for Vi IMproved.
It used to be Vi IMitation, but there are so many improvements that a name change was appropriate.
Vim is a text editor which includes almost all the commands from the Unix program "Vi" and a lot of new ones.
It is very useful for editing programs and other plain text.
All commands are given with the keyboard.
This has the advantage that you can keep your fingers on the keyboard and your eyes on the screen.
For those who want it, there is mouse support and a GUI version with scrollbars and menus (see |\verb!:h gui.txt!|).

An overview of this manual can be found in the file "help.txt", |\verb!:h help.txt!|.
It can be accessed from within Vim with the <Help> or <F1> key and with the |\verb!:help!| command (just type "\verb!:help!", without the bars or quotes).
The 'helpfile' option can be set to the name of the help file, in case it is not located in the default place.

Throughout this manual the differences between Vi and Vim are mentioned in curly braces, like this: \{Vi does not have on-line help\}.
See |\verb!:h vi_diff.txt!| for a summary of the differences between Vim and Vi.

This manual refers to Vim on various machines.
There may be small differences between different computers and terminals.
Besides the remarks given in this document, there is a separate document for each supported system, see |\verb!:h sys-file-list!|.

\phantomsection
\label{pronounce}
Vim is pronounced as one word, like Jim, not vi-ai-em.
It's written with a capital, since it's a name, again like Jim.

This manual is a reference for all the Vim commands and options.
This is not an introduction to the use of Vi or Vim, it gets a bit complicated here and there.
For beginners, there is a hands-on \hyperref[tutor]{|\texttt{tutor}|}.
To learn using Vim, read the user manual \ref{usr_toc}.

\phantomsection
\label{book}
There are many books on Vi that contain a section for beginners.
There are two books I can recommend:
\begin{itemize}

				\item
								\textit{Vim - Vi Improved} by Steve Oualline\newline
								This is the very first book completely dedicated to Vim.
								It is very good for beginners.
								The most often used commands are explained with pictures and examples.
								The less often used commands are also explained, the more advanced features are summarized.
								There is a comprehensive index and a quick reference.
								Parts of this book have been included in the user manual \hyperref[frombook]{|\texttt{frombook}|}.
								Published by New Riders Publishing.
								ISBN: 0735710015
								For more information try one of these:
								\begin{itemize}
												\item \url{http://iccf-holland.org/click5.html}
												\item \url{http://www.vim.org/iccf/click5.html}
								\end{itemize}

				\item
								\textit{Learning the Vi editor} by Linda Lamb and Arnold Robbins\newline
								This is a book about Vi that includes a chapter on Vim (in the sixth edition).
								The first steps in Vi are explained very well.
								The commands that Vim adds are only briefly mentioned.
								There is also a German translation.
								Published by O'Reilly.
								ISBN: 1-56592-426-6.
\end{itemize}

\subsection{Vim on the internet}
\label{internet}
\label{www}
\label{WWW}
\label{faq}
\label{FAQ}
\label{distribution}
\label{download}
The Vim pages contain the most recent information about Vim.
They also contain links to the most recent version of Vim.
The FAQ is a list of Frequently Asked Questions.
Read this if you have problems.

\begin{itemize}
				\item VIM home page: \url{http://www.vim.org/}
				\item VIM FAQ: \url{http://vimdoc.sf.net/}
				\item Downloading: \url{ftp://ftp.vim.org/pub/vim/MIRRORS}
\end{itemize}

\phantomsection
\label{new}
\label{usenet}
Usenet News group where Vim is discussed:\newline
\verb!comp.editors!\newline
This group is also for other editors.
If you write about Vim, don't forget to mention that.

\phantomsection
\label{mail-list}
\label{maillist}
There are several mailing lists for Vim:
\begin{itemize}
				\item 
								\verb!vim@vim.org!
								For discussions about using existing versions of Vim: Useful mappings, questions, answers, where to get a specific version, etc.
								There are quite a few people watching this list and answering questions, also for beginners.
								Don't hesitate to ask your question here.

				\item
								\label{vim-dev}
								\label{vimdev}
								\verb!vim-dev@vim.org!
								For discussions about changing Vim: New features, porting, patches, beta-test versions, etc.

				\item
								\label{vim-annouce}
								\verb!vim-announce@vim.org!
								Announcements about new versions of Vim; also for beta-test versions and ports to different systems.
								This is a read-only list.

				\item
								\label{vim-multibyte}
								\verb!vim-multibyte@vim.org!
								For discussions about using and improving the multi-byte aspects of Vim.

				\item
								\label{vim-mac}
								\verb!vim-mac@vim.org!
								For discussions about using and improving the Macintosh version of Vim.

				\item
								See \url{http://www.vim.org/maillist.php} for the latest information.
\end{itemize}

NOTE:
\begin{itemize}
				\item You can only send messages to these lists if you have subscribed!
				\item You need to send the messages from the same location as where you subscribed from (to avoid spam mail).
				\item Maximum message size is 40000 characters.
\end{itemize}

\phantomsection
\label{subscribe-maillist}
If you want to join, send a message to <vim-subscribe@vim.org>.
Make sure that your "From:" address is correct.
Then the list server will give you help on how to subscribe.


\phantomsection
\label{maillist-archive}
For more information and archives look on the Vim maillist page: \url{http://www.vim.org/maillist.php}. % I added this point and several others because it stressed me out

\subsubsection{Bug reports:}
\label{bugs}
\label{bug-reports}
\label{bugreport.vim}
Send bug reports to: Vim bugs \verb!bugs@vim.org!.
This is not a maillist but the message is redirected to the Vim maintainer.
Please be brief; all the time that is spent on answering mail is subtracted from the time that is spent on improving Vim!
Always give a reproducible example and try to find out which settings or other things influence the appearance of the bug.
Try different machines, if possible.
Send me patches if you can!

It will help to include information about the version of Vim you are using and your setup.
You can get the information with this command:
\begin{Verbatim}[samepage=true]
	 :so $VIMRUNTIME/bugreport.vim
\end{Verbatim}
This will create a file "bugreport.txt" in the current directory, with a lot of information of your environment.
Before sending this out, check if it doesn't contain any confidential information!

If Vim crashes, please try to find out where.
You can find help on this here: |\verb!:h debug.txt!|.

In case of doubt or when you wonder if the problem has already been fixed but
you can't find a fix for it, become a member of the vim-dev maillist and ask
your question there: \hyperref[maillist]{|\texttt{maillist}|}


\phantomsection
\label{year-2000}
\label{Y2K}
Since Vim internally doesn't use dates for editing, there is no year 2000 problem to worry about.
Vim does use the time in the form of seconds since January 1st 1970.
It is used for a time-stamp check of the edited file and the swap file, which is not critical and should only cause warning messages.

There might be a year 2038 problem, when the seconds don't fit in a 32 bit int anymore.
This depends on the compiler, libraries and operating system.
Specifically, \verb!time_t! and the \verb!ctime()! function are used.
And the \verb!time_t! is stored in four bytes in the swap file.
But that's only used for printing a file date/time for recovery, it will never affect normal editing.

The Vim \verb!strftime()! function directly uses the \verb!strftime()! system function.
\verb!localtime()! uses the \verb!time()! system function.
\verb!getftime()! uses the time
returned by the \verb!stat()! system function.
If your system libraries are year 2000 compliant, Vim is too.

The user may create scripts for Vim that use external commands.
These might introduce Y2K problems, but those are not really part of Vim itself.

\subsubsection{Credits}
\label{credits}
\label{author}
\label{Bram}
\label{Moolenaar}

Most of Vim was written by Bram Moolenaar <Bram@vim.org>. 

Parts of the documentation come from several Vi manuals, written by:
\begin{itemize}
				\item W.N. Joy
				\item Alan P.W. Hewett
				\item Mark Horton
\end{itemize}

The Vim editor is based on Stevie and includes (ideas from) other software, worked on by the people mentioned here.
Other people helped by sending me patches, suggestions and giving feedback about what is good and bad in Vim.

Vim would never have become what it is now, without the help of these people!
\begin{longtable}{c l}
				Ron Aaron & Win32 GUI changes\\
				Mohsin Ahmed & encryption\\
				Zoltan Arpadffy & work on VMS port\\
				Tony Andrews & Stevie\\
				Gert van Antwerpen & changes for DJGPP on MS-DOS\\
				Berkeley DB(3) & ideas for swap file implementation\\
				Keith Bostic & Nvi\\
				Walter Briscoe & Makefile updates, various patches\\
				Ralf Brown & SPAWNO library for MS-DOS\\
				Robert Colon & many useful remarks\\
				Marcin Dalecki & GTK+ GUI port, toolbar icons, gettext()\\
				Kayhan Demirel & sent me news in Uganda\\
				Chris \& John Downey & xvi (ideas for multi-windows version)\\
				Henk Elbers & first VMS port\\
				Daniel Elstner & GTK+ 2 port\\
				Eric Fischer & Mac port, 'cindent', and other improvements\\
				Benji Fisher & Answering lots of user questions\\
				Bill Foster & Athena GUI port\\
				Google & Lets me work on Vim one day a week\\
				Loic Grenie & xvim (ideas for multi windows version)\\
				Sven Guckes & Vim promoter and previous WWW page maintainer\\
				Darren Hiebert & Exuberant ctags\\
				Jason Hildebrand & GTK+ 2 port\\
				Bruce Hunsaker & improvements for VMS port\\
				Andy Kahn & Cscope support, GTK+ GUI port\\
				Oezguer Kesim & Maintainer of Vim Mailing Lists\\
				Axel Kielhorn & work on the Macintosh port\\
				Steve Kirkendall & Elvis\\
				Roger Knobbe & original port to Windows NT\\
				Sergey Laskavy & Vim's help from Moscow\\
				Felix von Leitner & Previous maintainer of Vim Mailing Lists\\
				David Leonard & Port of Python extensions to Unix\\
				Avner Lottem & Edit in right-to-left windows\\
				Flemming Madsen & X11 client-server, various features and patches\\
				Tony Mechelynck & answers many user questions\\
				Paul Moore & Python interface extensions, many patches\\
				Katsuhito Nagano & Work on multi-byte versions\\
				Sung-Hyun Nam & Work on multi-byte versions\\
				Vince Negri & Win32 GUI and generic console enhancements\\
				Steve Oualline & Author of the first Vim book \hyperref[frombook]{|\texttt{frombook}|}\\
				Dominique Pelle & valgrind reports and many fixes\\
				A.Politz & Many bug reports and some fixes\\
				George V. Reilly & Win32 port, Win32 GUI start-off\\
				Stephen Riehm & bug collector\\
				Stefan Roemer & various patches and help to users\\
				Ralf Schandl & IBM OS \& 390 port\\
				Olaf Seibert & DICE and BeBox version, regexp improvements\\
				Mortaza Shiran & Farsi patches\\
				Peter da Silva & termlib\\
				Paul Slootman & OS/2 port\\
				Henry Spencer & regular expressions\\
				Dany St-Amant & Macintosh port\\
				Tim Thompson & Stevie\\
				G. R. (Fred) Walter & Stevie\\
				Sven Verdoolaege & Perl interface\\
				Robert Webb & Command-line completion, GUI versions, and lots of patches\\
				Ingo Wilken & Tcl interface\\
				Mike Williams & PostScript printing\\
				Juergen Weigert & Lattice version, AUX improvements, UNIX and MS-DOS ports, autoconf\\
				Stefan 'Sec' Zehl & Maintainer of vim.org\\
\end{longtable}

I wish to thank all the people that sent me bug reports and suggestions.
The list is too long to mention them all here.
Vim would not be the same without the ideas from all these people: They keep Vim alive!

In this documentation there are several references to other versions of Vi:

\subsubsection{Vi}
\label{Vi}
\label{vi}
Vi "the original".
Without further remarks this is the version of Vi that appeared in Sun OS 4.x.
\verb!:version! returns "Version 3.7, 6/7/85".
Sometimes other versions are referred to.
Only runs under Unix.
Source code only available with a license.
More information on Vi can be found through: \url{http://vi-editor.org} [doesn't currently work...].

\subsubsection{Posix}
\label{Posix}
From the IEEE standard 1003.2, Part 2: Shell and utilities.
Generally known as "Posix".
This is a textual description of how Vi is supposed to work.
See |\verb!:h posix-compliance!|.

\subsubsection{Nvi}
\label{Nvi}
The "New" Vi.
The version of Vi that comes with BSD 4.4 and FreeBSD.
Very good compatibility with the original Vi, with a few extensions.
The version used is 1.79.
\verb!:version! returns "Version 1.79 (10/23/96)".
There has been no release the last few years, although there is a development version 1.81.
Source code is freely available.

\subsubsection{Elvis}
\label{Elvis}
Another Vi clone, made by Steve Kirkendall.
Very compact but isn't as flexible as Vim.*
The version used is 2.1.
It is still being developed.
Source code is freely available.

\subsection{Notation}
\label{notation}
When syntax highlighting is used to read this, text that is not typed literally is often highlighted with the Special group.
These are items in \verb![]!, \verb!{}! and \verb!<>!, and \verb!CTRL-X!.

Note that Vim uses all possible characters in commands.
Sometimes the \verb![]!, \verb!{}! and \verb!<>! are part of what you type, the context should make this clear.

\begin{description}
				\item[[]]
								Characters in square brackets are optional.

				\item[[count]]
								\label{count} \label{[count]}
								An optional number that may precede the command to multiply or iterate the command.
								If no number is given, a count of one is used, unless otherwise noted.
								Note that in this manual the [count] is not mentioned in the description of the command, but only in the explanation.
								This was done to make the commands easier to look up.
								If the 'showcmd' option is on, the (partially) entered count is shown at the bottom of the window.
								You can use <Del> to erase the last digit (|\verb!:h N<Del>!|).

				\item[["x]]
								\label{[quotex]}
								See |\verb!:h registers!|.
								An optional register designation where text can be stored.
								The x is a single character between 'a' and 'z' or 'A' and 'Z' or '"'', and in some cases (with the put command) between '0' and '9', '\%', '\#', or others.
								The uppercase and lowercase letter designate the same register, but the lowercase letter is used to overwrite the previous register contents, while the uppercase letter is used to append to the previous register contents.
								Without the ""x" or with """" the stored text is put into the unnamed register.

				\item[\{\}]
								\label{{}}
								Curly braces denote parts of the command which must appear, but which can take a number of different values.
								The differences between Vim and Vi are also given in curly braces (this will be clear from the context).

				\item[\{char1-char2\}]
								\label{{char1-char2}}
								A single character from the range char1 to char2.
								For example: {a-z} is a lowercase letter.
								Multiple ranges may be concatenated.
								For example, {a-zA-Z0-9} is any alphanumeric character.

				\item[\{motion\}]
								\label{{motion}}
								\label{movement}
								A command that moves the cursor.
								These are explained in |\verb!:h motion.txt!|.
								Examples:
								\begin{center}
												\begin{tabular}{c c}
																\verb!w! & to start of next word\\
																\verb!b! & to begin of current word\\
																\verb!4j! & four lines down\\
												\verb!/The<CR>! & to next occurrence of "The"\\
												\end{tabular}
								\end{center}
								This is used after an |\verb!:h operator!| command to move over the text that is to be operated upon.
								\begin{itemize}
												\item If the motion includes a count and the operator also has a count, the two counts are multiplied.
																For example: "2d3w" deletes six words.
												\item The motion can be backwards, e.g. "db" to delete to the start of the word.
												\item The motion can also be a mouse click.
																The mouse is not supported in every terminal though.
												\item The ":omap" command can be used to map characters while an operator is pending.
												\item Ex commands can be used to move the cursor.
																This can be used to call a function that does some complicated motion.
																The motion is always characterwise exclusive, no matter what ":" command is used.
																This means it's impossible to include the last character of a line without the line break (unless 'virtualedit' is set).
																If the Ex command changes the text before where the operator starts or jumps to another buffer the result is unpredictable.
																It is possible to change the text further down.
																Jumping to another buffer is possible if the current buffer is not unloaded.
								\end{itemize}

				\item[\{Visual\}]
								\label{{Visual}}
								A selected text area.
								It is started with the "v", "V", or CTRL-V command, then any cursor movement command can be used to change the end of the selected text.
								This is used before an |\verb!:h operator!| command to highlight the text that is to be operated upon.
								See |\verb!:h Visual-mode!|.

				\item[<character>]
								\label{<character>}
								A special character from the table below, optionally with modifiers, or a single ASCII character with modifiers.

				\item['c']
								\label{'character'}
								A single ASCII character.

				\item[CTRL-\{char\}]
								\label{CTRL-{char}}
								\{char\} typed as a control character; that is, typing {char} while holding the CTRL key down.
								The case of {char} does not matter; thus CTRL-A and CTRL-a are equivalent.
								But on some terminals, using the SHIFT key will produce another code, don't use it then.

				\item['option']
								\label{'option'}
								An option, or parameter, that can be set to a value, is enclosed in single quotes.
								See |\verb!:h options!|.

				\item["command"]
								\label{quotecommandquote}
								A reference to a command that you can type is enclosed in double quotes.

				\item[]
								\label{key-notation}
								\label{key-codes}
								\label{keycodes}
These names for keys are used in the documentation.
They can also be used with the "\verb!:map!" command (insert the key name by pressing CTRL-K and then the key you want the name for).

\begin{tabularx}{\textwidth}{|c|X|c|c|c|} % the 5 cm is completely arbitrary but it seems to work in the page
				\hline
				notation & meaning                   & equivalent & decimal             & value(s)\\ \hline
				<Nul>    & zero                      & CTRL-@     & 0 (stored as 10)    & \label{<Nul>}\\
				<BS>     & backspace                 & CTRL-H     & 8                   & \label{backspace}\\
				<Tab>    & tab                       & CTRL-I     & 9                   & \label{tab} \label{Tab}\\
				<NL>     & linefeed                  & CTRL-J     & 10 (used for <Nul>) & \label{linefeed}\\
				<FF>     & formfeed                  & CTRL-L     & 12                  & \label{formfeed}\\
				<CR>     & carriage return           & CTRL-M     & 13                  & \label{carriage-return}\\
				<Return> & same as <CR>              &            &                     & \label{<Return>}\\
				<Enter>  & same as <CR>              &            &                     & \label{<Enter>}\\
				<Esc>    & escape                    & CTRL-[     & 27                  & \label{escape} \label{<Esc>}\\
				<Space>  & space                     &            & 32                  & \label{space}\\
				<lt>     & less-than                 & <          & 60                  & \label{<lt>}\\
				<Bslash> & backslash                 & \          & 92                  & \label{backslash} \label{<Bslash>}\\
				<Bar>    & vertical bar              & |          & 124                 & \label{<Bar>}\\
				<Del>    & delete                    &            & 127\\
				<CSI>    & command sequence intro    & ALT-Esc    & 155                 & \label{<CSI>}\\
				<xCSI>   & CSI when typed in the GUI &            &                     & \label{<xCSI>}\\

				<EOL> & end-of-line (can be <CR>, <LF> or <CR><LF>, depends on system and 'fileformat') & & & \label{<EOL>}\\ %OK pour les &

				<Up>             & cursor-up                      &  &  & \label{cursor-up} \label{cursor_up}\\
				<Down>           & cursor-down                    &  &  & \label{cursor-down} \label{cursor_down}\\
				<Left>           & cursor-left                    &  &  & \label{cursor-left} \label{cursor_left}\\
				<Right>          & cursor-right                   &  &  & \label{cursor-right} \label{cursor_right}\\
				<S-Up>           & shift-cursor-up                &  &  & \\
				<S-Down>         & shift-cursor-down              &  &  & \\
				<S-Left>         & shift-cursor-left              &  &  & \\
				<S-Right>        & shift-cursor-right             &  &  & \\
				<C-Left>         & control-cursor-left            &  &  & \\
				<C-Right>        & control-cursor-right           &  &  & \\
				<F1> - <F12>     & function keys 1 to 12          &  &  & \label{function_key} \label{function-key}\\
				<S-F1> - <S-F12> & shift-function keys 1 to 12    &  &  & \label{<S-F1>}\\
				<Help>           & help key                       &  &  & \\
				<Undo>           & undo key                       &  &  & \\
				<Insert>         & insert key                     &  &  & \\
				<Home>           & home                           &  &  & \label{home}\\
				<End>            & end                            &  &  & \label{end}\\
				<PageUp>         & page-up                        &  &  & \label{page_up} \label{page-up}\\
				<PageDown>       & page-down                      &  &  & \label{page_down} \label{page-down}\\
				<kHome>          & keypad home (upper left)       &  &  & \label{keypad-home}\\
				<kEnd>           & keypad end (lower left)        &  &  & \label{keypad-end}\\
				<kPageUp>        & keypad page-up (upper right)   &  &  & \label{keypad-page-up}\\
				<kPageDown>      & keypad page-down (lower right) &  &  & \label{keypad-page-down}\\
				<kPlus>          & keypad +                       &  &  & \label{keypad-plus}\\
				<kMinus>         & keypad -                       &  &  & \label{keypad-minus}\\
				<kMultiply>      & keypad *                       &  &  & \label{keypad-multiply}\\
				<kDivide>        & keypad /                       &  &  & \label{keypad-divide}\\
				<kEnter>         & keypad Enter                   &  &  & \label{keypad-enter}\\
				<kPoint>         & keypad Decimal point           &  &  & \label{keypad-point}\\
				<k0> - <k9>      & keypad 0 to 9                  &  &  & \label{keypad-0} \label{keypad-9}\\
				<S-...>          & shift-key                      &  &  & \label{shift} \label{<S-}\\
				<C-...>          & control-key                    &  &  & \label{control} \label{ctrl} \label{<C-}\\
				<M-...>          & alt-key or meta-key            &  &  & \label{meta} \label{alt} \label{<M-}\\
				<A-...>          & same as <M-...>                &  &  & \label{<A-}\\
				<D-...>          & command-key (Macintosh only)   &  &  & \label{<D-}\\
				<t\_xx>          & key with "xx" entry in termcap &  &  & \\
				\hline
\end{tabularx}

\textit{Note}: The shifted cursor keys, the help key, and the undo key are only available on a few terminals.
On the Amiga, shifted function key 10 produces a code (CSI) that is also used by key sequences.
It will be recognized only after typing another key.

\textit{Note}: There are two codes for the delete key.
127 is the decimal ASCII value for the delete key, which is always recognized.
Some delete keys send another value, in which case this value is obtained from the termcap entry "kD".
Both values have the same effect.
Also see |\verb!:h :fixdel!|.

\textit{Note}: The keypad keys are used in the same way as the corresponding "normal" keys.
For example, <kHome> has the same effect as <Home>.
If a keypad key sends the same raw key code as its non-keypad equivalent, it will be recognized as the non-keypad code.
For example, when <kHome> sends the same code as <Home>, when pressing <kHome> Vim will think <Home> was pressed.
Mapping <kHome> will not work then.

\phantomsection
\label{<>}
Examples are often given in the <> notation.
Sometimes this is just to make clear what you need to type, but often it can be typed literally, e.g., with the "\verb!:map!" command.
The rules are:
\begin{enumerate}
				\item Any printable characters are typed directly, except backslash and '<'
				\item A backslash is represented with "\textbackslash\textbackslash", double backslash, or "<Bslash>".
				\item A real '<' is represented with "\textbackslash<" or "<lt>".  When there is no
								confusion possible, a '<' can be used directly.
				\item "<key>" means the special key typed.  This is the notation explained in
								the table above.  A few examples:
								\begin{center}
												\begin{tabular}{c c}
																<Esc> & Escape key\\
																<C-G> & CTRL-G\\
																<Up> & cursor up key\\
																<C-LeftMouse> & Control- left mouse click\\
																<S-F11> & Shifted function key 11\\
																<M-a> & Meta- a  ('a' with bit 8 set)\\
																<M-A> & Meta- A  ('A' with bit 8 set)\\
																<t\_kd> & "kd" termcap entry (cursor down key)\\
												\end{tabular}
								\end{center}
\end{enumerate}
\end{description}

If you want to use the full <> notation in Vim, you have to make sure the '<' flag is excluded from 'cpoptions' (when 'compatible' is not set, it already is by default).
\begin{Verbatim}[samepage=true]
 :set cpo-=<
\end{Verbatim}
The <> notation uses <lt> to escape the special meaning of key names.
Using a backslash also works, but only when 'cpoptions' does not include the 'B' flag.

Examples for mapping CTRL-H to the six characters "<Home>":
\begin{Verbatim}[samepage=true]
 :imap <C-H> \<Home>
 :imap <C-H> <lt>Home>
\end{Verbatim}
The first one only works when the 'B' flag is not in 'cpoptions'.
The second one always works.
To get a literal "<lt>" in a mapping:
\begin{Verbatim}[samepage=true]
 :map <C-L> <lt>lt>
\end{Verbatim}

For mapping, abbreviation and menu commands you can then copy-paste the examples and use them directly.
Or type them literally, including the '<' and '>' characters.
This does NOT work for other commands, like "\verb!:set!" and "\verb!:autocmd!"!

\subsection{Modes, introduction}
\label{vim-modes-intro}
\label{vim-modes}
Vim has six BASIC modes:

\subsubsection{Normal Mode}
\label{Normal}
\label{Normal-mode}
\label{command-mode}
In Normal mode you can enter all the normal editor commands.
If you start the editor you are in this mode (unless you have set the 'insertmode' option, see below).
This is also known as command mode.

\subsubsection{Visual mode}
This is like Normal mode, but the movement commands extend a highlighted area.
When a non-movement command is used, it is executed for the highlighted area.
See |\verb!:h Visual-mode!|.
If the 'showmode' option is on \verb!-- VISUAL --! is shown
at the bottom of the window.

\subsubsection{Select mode}
This looks most like the MS-Windows selection mode.
Typing a printable character deletes the selection and starts Insert mode.
See |\verb!:h Select-mode!|.
If the 'showmode' option is on \verb!-- SELECT --! is shown at the bottom of the window.

\subsubsection{Insert mode}
In Insert mode the text you type is inserted into the buffer.
See |\verb!:h Insert-mode!|.
If the 'showmode' option is on \verb!-- INSERT --! is shown at the bottom of the window.

\subsubsection{Command-line mode, Cmdlinemode}
In Command-line mode (also called Cmdline mode) you can enter one line of text at the bottom of the window.
This is for the Ex commands, ":", the pattern search commands, "?" and "/", and the filter command, "!".
|\verb!:h Cmdline-mode!|

\subsubsection{Ex mode}
Like Command-line mode, but after entering a command you remain in Ex mode.
Very limited editing of the command line.
\hyperref[Ex-mode]{|\texttt{Ex-mode}|}

There are six ADDITIONAL modes.  These are variants of the BASIC modes:


\subsubsection{Operator-pending mode}
\label{Operator-pending}
\label{Operator-pending-mode}
This is like Normal mode, but after an operator command has started, and Vim is waiting for a {motion} to specify the text that the operator will work on.

\subsubsection{Replace mode}
Replace mode is a special case of Insert mode.
You can do the same things as in Insert mode, but for each character you enter, one character of the existing text is deleted.
See |\verb!:h Replace-mode!|.
If the 'showmode' option is on \verb!-- REPLACE --! is shown at the bottom of the window.

\subsubsection{Virtual Replace mode}
Virtual Replace mode is similar to Replace mode, but instead of file characters you are replacing screen real estate.
See |\verb!:h Virtual-Replace-mode!|.
If the 'showmode' option is on \verb!-- VREPLACE --! is shown at the bottom of the window.

\subsubsection{Insert Normal mode}
Entered when CTRL-O given in Insert mode.
This is like Normal mode, but after executing one command Vim returns to Insert mode.
If the 'showmode' option is on \verb!-- (insert) --! is shown at the bottom of the window.

\subsubsection{Insert Visual mode}
Entered when starting a Visual selection from Insert mode, e.g., by using CTRL-O and then "v", "V" or CTRL-V.
When the Visual selection ends, Vim returns to Insert mode.
If the 'showmode' option is on \verb!-- (insert) VISUAL --! is shown at the bottom of the window.

\subsubsection{Insert Select mode}
Entered when starting Select mode from Insert mode.
E.g., by dragging the mouse or <S-Right>.
When the Select mode ends, Vim returns to Insert mode.
If the 'showmode' option is on \verb!-- (insert) SELECT --!
is shown at the bottom of the window.

\subsection{Switching from mode to mode}
\label{mode-switching}

If for any reason you do not know which mode you are in, you can always get back to Normal mode by typing <Esc> twice.
This doesn't work for Ex mode though, use ":visual".
You will know you are back in Normal mode when you see the screen flash or hear the bell after you type <Esc>.
However, when pressing <Esc> after using CTRL-O in Insert mode you get a beep but you are still in Insert mode, type <Esc> again.

\phantomsection
\label{i_esc}
\begin{tabular}{c c c c c c c c}
				\backslashbox{FROM mode}{TO mode} & Normal & Visual                & Select & Insert   & Replace  & Cmd-line & Ex \\
				Normal                        &        & v       V       \textasciicircum V    & *4     & *1       & R gR     & : / ? !  & Q\\
				Visual                        & *2     &                       & \textasciicircum G     & c      C & --       & :        & --\\
				Select           *5     & \textasciicircum O                 \textasciicircum G &        & *6       & --       & --       & --\\
				Insert                            & <Esc>  & --                    & --     &          & <Insert> & --       & --\\
				Replace                           & <Esc>  & --                    & --     & <Insert> &          & --       & --\\
				Command-line                      & *3     & --                    & --     & :start   & --       &          & --\\
				Ex                                & :vi    & --                    & --     & --       & --       & --       & \\
\end{tabular}

-- not possible

\begin{itemize}
				\item *1 Go from Normal mode to Insert mode by giving the command "i", "I", "a", "A", "o", "O", "c", "C", "s" or "S".
				\item *2 Go from Visual mode to Normal mode by giving a non-movement command, which causes the command to be executed, or by hitting <Esc> "v", "V" or "CTRL-V" (see |\verb!:h v_v!|), which just stops Visual mode without side effects.
				\item *3 Go from Command-line mode to Normal mode by:
								\begin{itemize}
												\item Hitting <CR> or <NL>, which causes the entered command to be executed.
												\item Deleting the complete line (e.g., with CTRL-U) and giving a final <BS>.
												\item Hitting CTRL-C or <Esc>, which quits the command-line without executing
																the command.
								\end{itemize}
								In the last case <Esc> may be the character defined with the 'wildchar' option, in which case it will start command-line completion.
								You can ignore that and type <Esc> again.
								\{Vi: when hitting <Esc> the command-line is executed.
								This is unexpected for most people; therefore it was changed in Vim.
								But when the <Esc> is part of a mapping, the command-line is executed.
								If you want the Vi behaviour also when typing <Esc>, use "\verb!:cmap ^V<Esc> ^V^M!"\}
				\item *4 Go from Normal to Select mode by:
								\begin{itemize}
												\item use the mouse to select text while 'selectmode' contains "mouse"
												\item use a non-printable command to move the cursor while keeping the Shift key pressed, and the 'selectmode' option contains "key"
												\item use "v", "V" or "CTRL-V" while 'selectmode' contains "cmd"
												\item use "gh", "gH" or "g CTRL-H"  |\verb!:h g_CTRL-H!|
								\end{itemize}
				\item *5 Go from Select mode to Normal mode by using a non-printable command to move the cursor, without keeping the Shift key pressed.
				\item *6 Go from Select mode to Insert mode by typing a printable character.
								The selection is deleted and the character is inserted.
\end{itemize}

If the 'insertmode' option is on, editing a file will start in Insert mode.

\phantomsection
\label{CTRL-backslash_CTRL-N}
\label{i_CTRL-backslash_CTRL-N}
\label{c_CTRL-backslash_CTRL-N}
\label{v_CTRL-backslash_CTRL-N}
Additionally the command CTRL-\textbackslash CTRL-N or <C-\textbackslash><C-N> can be used to go to Normal mode from any other mode.
This can be used to make sure Vim is in Normal mode, without causing a beep like <Esc> would.
However, this does not work in Ex mode.
When used after a command that takes an argument, such as |f| or |\verb!:h m!|, the timeout set with \verb!ttimeoutlen! applies.

\phantomsection
\label{CTRL-backslash_CTRL-G}
\label{i_CTRL-backslash_CTRL-G}
\label{c_CTRL-backslash_CTRL-G}
\label{v_CTRL-backslash_CTRL-G}
The command CTRL-\textbackslash CTRL-G or <C-\textbackslash><C-G> can be used to go to Insert mode when 'insertmode' is set.
Otherwise it goes to Normal mode.
This can be used to make sure Vim is in the mode indicated by 'insertmode', without knowing in what mode Vim currently is.

\subsubsection{Q}
\label{Q}
\label{mode-Ex}
\label{Ex-mode}
\label{Ex}
\label{EX}
\label{E501}
Switch to "Ex" mode.
This is a bit like typing ":" commands one after another, except:
\begin{itemize}
				\item You don't have to keep pressing ":".
				\item The screen doesn't get updated after each command.
				\item There is no normal command-line editing.
				\item Mappings and abbreviations are not used.
\end{itemize}
In fact, you are editing the lines with the "standard" line-input editing commands (<Del> or <BS> to erase, CTRL-U to kill the whole line).
Vim will enter this mode by default if it's invoked as "ex" on the command-line.
Use the "\verb!:vi!" command |\verb!:h :visual!| to exit "Ex" mode.
Note: In older versions of Vim "Q" formatted text, that is now done with |\verb!:h gq!|.
But if you use the
|\verb!:h vimrc_example.vim!| script "Q" works like "gq".

\subsubsection{gQ}
\label{gQ}
Switch to "Ex" mode like with "Q", but really behave like typing ":" commands after another.
All command line editing, completion etc. is available.
Use the "\verb!:vi!" command |\verb!:h :visual!| to exit "Ex" mode.
\{not in Vi\}

\subsection{The window contents}
\label{window-contents}
In Normal mode and Insert/Replace mode the screen window will show the current contents of the buffer: What You See Is What You Get.
There are two exceptions:
\begin{itemize}
				\item When the 'cpoptions' option contains \verb!$!,  and the change is within one line, the text is not directly deleted, but a \verb!$ is! put at the last deleted character.
				\item When inserting text in one window, other windows on the same text are not updated until the insert is finished.
\end{itemize}
\{Vi: The screen is not always updated on slow terminals\}

Lines longer than the window width will wrap, unless the 'wrap' option is off (see below).
The 'linebreak' option can be set to wrap at a blank character.

If the window has room after the last line of the buffer, Vim will show \verb!~! in the first column of the last lines in the window, like this:
\begin{Verbatim}[samepage=true]
		+-----------------------+
		|some line              |
		|last line              |
		|~                      |
		|~                      |
		+-----------------------+
\end{Verbatim}

Thus the \verb!~! lines indicate that the end of the buffer was reached.

If the last line in a window doesn't fit, Vim will indicate this with a \verb'@' in the first column of the last lines in the window, like this:

\begin{Verbatim}[samepage=true]
		+-----------------------+
		|first line             |
		|second line            |
		|@                      |
		|@                      |
		+-----------------------+
\end{Verbatim}

Thus the '@' lines indicate that there is a line that doesn't fit in the window.

When the "lastline" flag is present in the 'display' option, you will not see \verb'@' characters at the left side of window.
If the last line doesn't fit completely, only the part that fits is shown, and the last three characters of the last line are replaced with "@@@", like this:

\begin{Verbatim}[samepage=true]
		+-----------------------+
		|first line             |
		|second line            |
		|a very long line that d|
		|oesn't fit in the wi@@@|
		+-----------------------+
\end{Verbatim}

If there is a single line that is too long to fit in the window, this is a special situation.
Vim will show only part of the line, around where the cursor is.
There are no special characters shown, so that you can edit all parts of this line.
\{Vi: gives an "internal error" on lines that do not fit in the window\}

The '@' occasion in the 'highlight' option can be used to set special highlighting for the '@' and \verb'~' characters.
This makes it possible to distinguish them from real characters in the buffer.

The 'showbreak' option contains the string to put in front of wrapped lines.

\phantomsection
\label{wrap-off}
If the 'wrap' option is off, long lines will not wrap.
Only the part that fits on the screen is shown.
If the cursor is moved to a part of the line that is not shown, the screen is scrolled horizontally.
The advantage of this method is that columns are shown as they are and lines that cannot fit on the screen can be edited.
The disadvantage is that you cannot see all the characters of a line at once.
The 'sidescroll' option can be set to the minimal number of columns to scroll.
\{Vi: has no 'wrap' option\}

All normal ASCII characters are displayed directly on the screen.
The <Tab> is replaced with the number of spaces that it represents.
Other non-printing characters are replaced with "\textasciicircum {char}", where {char} is the non-printing character with 64 added.
Thus character 7 (bell) will be shown as "\textasciicircum G".
Characters between 127 and 160 are replaced with "\textasciitilde \{char\}", where \{char\} is the character with 64 subtracted.
These characters occupy more than one position on the screen.
The cursor can only be positioned on the first one.

 If you set the 'number' option, all lines will be preceded with their
 number.  Tip: If you don't like wrapping lines to mix with the line numbers,
 set the 'showbreak' option to eight spaces:
\begin{Verbatim}[samepage=true]
		:set showbreak=\ \ \ \ \ \ \ \ 
\end{Verbatim}

If you set the 'list' option, <Tab> characters will not be shown as several spaces, but as "\textasciicircum I".
A \verb!$! will be placed at the end of the line, so you can find trailing blanks.

In Command-line mode only the command-line itself is shown correctly.
The display of the buffer contents is updated as soon as you go back to Command mode.

The last line of the window is used for status and other messages.
The status messages will only be used if an option is on:
\begin{center}
				\begin{tabular}{|c|c|c|c|}
								\hline
								status message     & option     & default & Unix default\\ \hline
								current mode       & 'showmode' & on      & on\\ \hline
								command characters & 'showcmd'  & on      & off\\ \hline
								cursor position    & 'ruler'    & off     & off\\ \hline
				\end{tabular}
\end{center}

The current mode is \verb"-- INSERT --" or \verb"-- REPLACE --", see |\verb!:h 'showmode'!|.
The command characters are those that you typed but were not used yet.
\{Vi: does not show the characters you typed or the cursor position\}

If you have a slow terminal you can switch off the status messages to speed up editing:
		\begin{verbatim}
		:set nosc noru nosm
		\end{verbatim}

If there is an error, an error message will be shown for at least one second (in reverse video).
\{Vi: error messages may be overwritten with other messages before you have a chance to read them\}

Some commands show how many lines were affected.
Above which threshold this happens can be controlled with the 'report' option (default 2).

On the Amiga Vim will run in a CLI window.
The name Vim and the full name of the current file name will be shown in the title bar.
When the window is resized, Vim will automatically redraw the window.
You may make the window as small as you like, but if it gets too small not a single line will fit in it.
Make it at least 40 characters wide to be able to read most messages on the last line.

On most Unix systems, resizing the window is recognized and handled correctly by Vim.
\{Vi: not ok\}

\subsection{Definitions}
\label{definitions}
\begin{tabularx}{\textwidth}{c X}
				screen & The whole area that Vim uses to work in.
				This can be a terminal emulator window.
				Also called "the Vim window". \\
				window & A view on a buffer.
\end{tabularx}

A screen contains one or more windows, separated by status lines and with the command line at the bottom.

% This ASCII art doesn't look right in the editor but looks OK in the pdf.
\begin{Verbatim}[samepage=true] 
				+-------------------------------+
			screen  | window 1   | window 2         |
				|            |                  |
				|            |                  |
				|= status line =|= status line =|
				| window 3                      |
				|                               |
				|                               |
				|==== status line ==============|
				|command line                   |
				+-------------------------------+
\end{Verbatim}

The command line is also used for messages.
It scrolls up the screen when there is not enough room in the command line.

A difference is made between four types of lines:

\begin{center}
				\begin{tabularx}{\textwidth}{c X}
								buffer lines & 
								The lines in the buffer.
								This is the same as the lines as they are read from/written to a file.
								They can be thousands of characters long. \\

								logical lines & 
								The buffer lines with folding applied.
								Buffer lines in a closed fold are changed to a single logical line: "+-- 99 lines folded".
								They can be thousands of characters long.\\

								window lines & 
								The lines displayed in a window: A range of logical lines with wrapping, line breaks, etc.  applied.
								They can only be as long as the width of the window allows, longer lines are wrapped or truncated.\\

								screen lines & 
								The lines of the screen that Vim uses.
								Consists of the window lines of all windows, with status lines and the command line added.
								They can only be as long as the width of the screen allows.
								When the command line gets longer it wraps and lines are scrolled to make room.
				\end{tabularx}
\end{center}

\begin{center}
				\begin{tabular}{|c|c|c|c|}
								buffer lines    & logical lines  & window lines       & screen lines \\ \hline
								1. one          & 1. one         & 1. +-- folded      & 1.  +-- folded\\
								2. two          & 2. +-- folded  & 2. five            & 2.  five\\
								3. three        & 3. five        & 3. six             & 3.  six\\
								4. four         & 4. six         & 4. seven           & 4.  seven\\
								5. five         & 5. seven       &                    & 5.  === status line ===\\
								6. six          &                &                    & 6.  aaa\\
								7. seven        &                &                    & 7.  bbb\\
																&                &                    & 8.  ccc ccc c\\
								1. aaa          & 1. aaa         & 1. aaa             & 9.  cc\\
								2. bbb          & 2. bbb         & 2. bbb             & 10. ddd\\
								3. ccc ccc ccc  & 3. ccc ccc ccc & 3. ccc ccc c       & 11. \textasciitilde \\
								4. ddd          & 4. ddd         & 4. cc              & 12. === status line ===\\
																&                & 5. ddd             & 13. (command line)\\
																&                & 6. \textasciitilde & \\
				\end{tabular}
\end{center}
\clearpage
